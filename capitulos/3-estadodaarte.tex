\chapter{Trabalhos Correlatos}\label{cap:estArte}

\section{Considerações Iniciais}\label{sec:primTrab}

O processo de extração de conhecimento válido acerca de bases de dados abertas públicas, seja pela ótica da ciência de dados ou análise de dados tradicional, apresenta-se como uma importante linha de pesquisa tanto na área da computação aplicada quanto nos demais campos interdisciplinares. Com isso, diversos trabalhos referentes a essa temática objetivam, de forma geral, compreender aspectos econômicos e sociais ou propostas governamentais baseando-se em simulações de cenários, análises estatísticas, aplicação de modelos de inteligência computacional e observações críticas acerca dos resultados, utilizando como objeto de estudos os dados relacionados a problemática abordada. 

Diante disso, o presente capítulo apresenta um panorama geral dos estudos correlacionados a temática abordada nessa dissertação, explicitando os trabalhos julgados mais relevantes perante a problemática descrita no escopo desta pesquisa. Dessa forma, a Tabela~\ref{tab:cap03:correlatos} apresenta uma síntese acerca desses trabalhos correlatos, mostrando de maneira concisa as considerações realizadas acerca dos mesmos. 


\section{Correlatos}

A aplicação da ciência ou análise de dados em dados abertos públicos apresenta-se promissora, e difusa, em meio aos diversos segmentos governamentais. Estudos utilizando os microdados da economia, educação e saúde pública, por exemplo, tendem a apresentar informações relevantes para a compreensão de alguns fenômenos sociais e podem servir como subsídio para auxiliar tomadas de decisões em gestão pública \cite{cap03_ref1, cap03_ref2}. Avaliando o estado da arte da previdência social brasileira, eixo temático desta pesquisa, evidenciam-se, no geral, trabalhos que objetivam compreender e discutir a sua contribuição social e a transparência dos modelos propostos pelo governo relativos à sustentabilidade do sistema.

Dessa forma, acerca dos impactos socioeconômicos causados pela previdência social nacional, o trabalho de \cite{cap02_ref22} utilizou dados da PNAD de 2001 a 2007 para verificar quais variáveis de rendimento influenciam na redução ou aumento da desigualdade de distribuição renda – mensurada através da decomposição do índice de Gini, semelhante ao demonstrado na subseção~\ref{cap:referencias:gini} desta dissertação. Os resultados demonstraram que a parcela de aposentadorias e pensões, embora tenham colaborado para a redução do coeficiente de Gini com alterações ocorridas no decorrer dos anos, apresentava regressividade, ou seja, colaboram para um aumento na concentração de renda. Além disso, destaca-se que embora o trabalho não seja recente, o mesmo apresentou importantes discussões que colaboraram para um melhor compreendimento da problemática abordada nessa pesquisa.

O trabalho de \cite{cap04_ref11}, semelhante ao anterior, avalia a participação que os benefícios de aposentadorias e pensões, oficiais e não oficiais, apresentam na desigualdade de rendimento dos domicílios brasileiros. A autora utiliza os dados da Pesquisa Nacional por Amostra de Domicílios (PNAD) de 2004 a 2017 e a metodologia de decomposição do índice de Gini, aplicando-os no Brasil e suas regiões, semelhante ao trabalho anterior. No entanto, seus resultados demonstraram que esses benefícios apresentam características diferentes quando divididos em categorias de salários, ou seja, as aposentadorias e pensões até um salário mínimo colaboram para a desconcentração de renda, enquanto as acima de um salário mínimo apresentavam característica oposta, contribuindo para aumentar o coeficiente de Gini. 

Em \cite{cap03_ref3} foram utilizados os dados censitários, disponibilizados pelo Instituto Paranaense de Desenvolvimento Econômico e Social (IPARDES), dos anos de 1980, 1991, 2000 e 2010 para os 399 municípios paranaenses, e microdados de rendimento presentes nas PNADs de 1988 a 2012, objetivando compreender o processo de envelhecimento da população no estado e seus impactos socioeconômicos na distribuição de renda da população. O estudo demonstrou que no intervalo de 30 anos avaliado, a proporção de pessoas acima de 60 anos mais que triplicou, provocando um aumento paralelo no percentual de participação dos benefícios de aposentadorias e pensões na renda total, apontando, dessa forma, uma preocupação na formação de políticas públicas ao mercado de trabalho, sistema de previdência, saúde e lazer para o público analisado.

A pesquisa de \cite{cap03_ref4}, por sua vez, propôs um modelo econométrico para compreender como a previdência social atua sobre as distribuições de renda regionais e estaduais no Brasil. Os resultados demonstraram que, em 2010, o sistema previdenciário foi um instrumento eficaz para a melhoria na distribuição de renda regional, mensurada através do índice de Gini, a qual apresentou impactos positivos diretos no PIB per capita brasileiro. 

Em \cite{cap04_ref12} foram utilizados dados municipais relacionados a pagamentos de benefícios de aposentadorias, arrecadações e fundo de participação municipal, de 2017, para compreender o cenário previdenciário nacional. Mediante a isso, o estudo destaca a Previdência Social como principal instrumento assistencial para o combate à pobreza, relatando que, sem os benefícios pagos mensalmente aos pensionistas e aposentados, diversas áreas rurais dos municípios de pequeno porte estariam em situação de calamidade social.

Por outro lado, a literatura apresenta diversos trabalhos relacionados à sustentabilidade do sistema previdenciário brasileiro e à transparência das políticas públicas que tangenciam essa temática. Com isso, o trabalho de  \cite{cap03_ref5}, por exemplo, buscou quantificar, usando um modelo de previsão a longo prazo, o impacto da demografia sobre as despesas da previdência social como proporção do Produto interno Bruto (PIB) para o Brasil. A pesquisa utilizou um modelo simples de projeção, demostrando que em 2060 as despesas do sistema compreenderiam aproximadamente 19\% do PIB nacional. Entretanto, apesar da grande relevância dos resultados, o modelo é útil apenas para ilustrar os efeitos do envelhecimento da população em relação às despesas previdenciárias, não sendo apropriado para investigações acerca de propostas de reforma. 

Em \cite{cap05_ref9}, os autores, motivados pelas discussões políticas acerca da reforma da previdência prevista na PEC 287/16, investigaram os resultados divulgados pelo governo federal, os quais foram utilizados para justificar a proposta de política pública. O trabalho apontou que os dados oficiais apresentavam projeções tendenciosas a curto prazo e com erros elevados a longo prazo, uma vez que tais estimativas não consideravam intervalos de confiança. Além disso, a pesquisa relatou a impossibilidade de reprodução dos resultados divulgados pela Lei de Diretrizes Orçamentarias (LDOs), devido, principalmente, à falta de transparência nesses documentos oficiais, tanto na sua metodologia quanto nos dados utilizados.

A pesquisa de \cite{cap03_ref6} apresenta um panorama detalhado referente a atual situação da Previdência Social brasileira, elencando os principais pontos previstos pela PEC 287/16 e relatando os fatores econômicos utilizados para justifica-la. Os autores classificam a proposta citada como imprescindível para a garantia da sustentabilidade fiscal do país, uma vez que o acelerado envelhecimento populacional eleva os custos excessivos de financiamento para as gerações futuras. Semelhantemente, o trabalho de \cite{cap03_ref7} afirma que na ausência de uma reforma, será praticamente impossível respeitar o teto do gasto público até o ano de 2026, último ano de vigência antes de uma possível revisão. Diante disso, o estudo discute a PEC 287/16 e outras duas reformas, sendo uma elaborada pelos autores, demonstrando que, caso aprovado, um reajuste nas diretrizes do sistema previdenciário possibilitaria um controle no teto de gastos, proporcionando condições para a continuidade e crescimento do programa.

Por outro lado, em \cite{cap05_ref10} foram identificadas falsificações, ou possíveis erros, nos resultados utilizados pelo governo federal para justificar a reforma previdenciária PEC 06/2019. Utilizando as planilhas oficiais do Ministério da Economia referentes a Reforma da Previdência, obtidas através da LAI, foram refeitos os cálculos, demonstrando que os benefícios de aposentadorias dos trabalhadores mais pobre diminuem com a reforma proposta. Além disso, notou-se que as aposentadorias obtidas por tempo de contribuição – principal alvo da reforma –, pelas regras atuais, com idades mais novas, financiam os benefícios de menor valor dos trabalhadores que se aposentam com idade mais avançada, fator que, possivelmente, colabora para redução da desigualdade social.

É importante destacar que, embora tenham participação direta na construção desta dissertação, nenhum dos trabalhos citados acima investigou a participação que as aposentadorias e pensões têm na distribuição de renda municipal do Brasil, e nem os efeitos que a uma reforma previdenciária causaria na concessão desses benefícios, principais objetivos desta dissertação. Dessa forma, objetivando esclarecer de forma resumida a abordagem de cada pesquisa, a Tabela~\ref{tab:cap03:correlatos} apresenta uma síntese dos trabalhos correlatos a esta pesquisa ressaltando as principais lacunas identificadas.

\begin{longtable}{|p{0.25\textwidth}|p{0.35\textwidth}|p{0.35\textwidth}|}
\caption{Síntese dos principais trabalhos correlatos.}
\\\hline 
\textbf{Referência} & \textbf{Objetivos} & \textbf{Lacunas} \\ \hline
\footnotesize{(HOFFMANN, 2009)} & Avaliar o impacto que as aposentadorias e pensões causam na distribuição de renda do Brasil entre 2001 e 2007 & Concentra-se em uma avaliação nacional, não abordando unidades territoriais menores \\ \hline
\footnotesize{(NAKATANI-MACEDO, 2015)} & Avaliar o impacto que as aposentadorias e pensões causam na distribuição de renda do Brasil e regiões entre 2004 e 2014 & Concentra-se em uma avaliação nacional e estadual, não abordando unidades territoriais menores \\ \hline
\footnotesize{(NAKATANI-MACEDO et al., 2015)} & Avaliar o impacto que as aposentadorias e pensões causam na distribuição de renda do estado do Paraná e regiões entre 1980 e 2010 & Concentra-se em uma avaliação isolada de apenas um estado brasileiro \\ \hline
\footnotesize{(BARBOSA, 2015)} & Propor um modelo econométrico que compreenda como as aposentadorias implicam na distribuição de renda das regiões e estados brasileiros & Semelhante aos anteriores, não aplica uma avaliação acerca de unidades territoriais menores \\ \hline
\footnotesize{(SOLÓN, 2019)} & Utiliza dados previdenciários municipais de 2017 para avaliar a situação do sistema em cada cidade & Não estima o impacto que os benefícios causam na distribuição de renda da população de cada município \\ \hline
\footnotesize{(COSTANZI; ANSILIERO, 2017)} & Aplica um modelo de previsão a longo prazo para avaliar a a situação da Previdência Social em 2060 & Não considera os impactos sociais que uma reforma provocaria e suas projeções não apresentam intervalos de confiança \\ \hline
\footnotesize{(SILVA et al., 2017)} & Recalcular as projeção realizadas pelo governo para verificar se os argumentos utilizadas para justificar  a PEC 287/16 eram válidos & Não avaliou os possíveis impactos sociais que reforma a reforma proposta causaria \\ \hline
\footnotesize{(COSTANZI et al., 2018)} & Elencar os principais pontos da PEC 287/16 mediante a atual situação do sistema previdenciário nacional & Não pondera os possíveis pontos negativos que a proposta acarretaria no cenário social dos trabalhadores \\ \hline
\footnotesize{(GIAMBIAGI;PINTO; ROTHMULLER, 2018)} & Avaliar a PEC 287/16, e outras duas reformas, como alternativas para controlar os gastos públicos do sistema previdenciário nacional & Semelhante ao anterior, desconsidera os impactos sociais que as propostas poderiam causar \\ \hline
\footnotesize{(BASTOS et al., 2019)} & Recalcular os resultados divulgados pelo governo utilizados para justificar a reforma da previdência correspondente a PEC 06/2019 & O estudo, embora enfatize os aspectos sociais negativos da reforma, não apresenta estimativas acerca do cenário previdenciário \\ \hline
%\end{tabular}
\caption*{\footnotesize{Elaborado pelo autor.}}
\label{tab:cap03:correlatos}
\end{longtable}

\section{Considerações Finais}
Este capítulo apresentou os trabalhos correlatos que nortearam a realização dessa pesquisa. Avaliando o estado da arte da problemática abordada nessa dissertação, observa-se um conjunto de estudos referente à análise dos impactos socioeconômicos gerados pelo sistema previdenciário brasileiro, e à avaliação da sustentabilidade de propostas governamentais. Todavia, destaca-se que grande parte desses estudos não aborda nas suas metodologias as aplicações de técnicas, ou estratégias, de análise de dados, normalmente utilizadas para a abstração dos resultados discutidos.

Nesse contexto, os estudos de caso propostos por essa pesquisa realizam uma abordagem diferente da comumente encontrada na literatura. Além da avaliação crítica realizada acerca das análises propostas, será detalhado o procedimento metodológico utilizado - desde a extração dos dados sem representatividade, até a abstração de conhecimento válido acerca dos resultados gerados. Tal abordagem apresenta-se promissora, dentre outros fatores, por possibilitar a replicação do trabalho pela comunidade científica, proporcionando, dessa forma, o desenvolvimento de uma pesquisa colaborativa mais eficiente.

