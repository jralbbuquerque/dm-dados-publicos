\chapter{Considerações Finais}\label{cap:conclusao}

A população brasileira está diante de uma possível reforma previdenciária que causará impactos diretos na vida de todos os trabalhadores. Logo, é de extrema importância que a sociedade compreenda as razões que motivaram o governo propor tal medida e os efeitos que tais alterações tendem a causar no cenário social. Paralelamente, é possível observar que o advento da ciência de dados aliada à disseminação de dados abertos governamentais, apresentam-se como importantes ferramentas para a realização de análises cada vez mais complexas e precisas acerca da problemática descrita.

Embora tal abordagem apresente-se em evidência na literatura, destaca-se a ausência de trabalhos que analisem os impactos socioeconômicos relacionados a temática abordada. Isto é, investigando o estado da arte, observou-se uma gama de pesquisas que avaliaram as participações de aposentadorias e pensões na distribuição de renda em um caráter nacional - impossibilitando interpretações em unidades territoriais menores -, e estudos que investigaram os impactos macroeconômicos que a reforma da previdência poderia causar.

Diante disso, esta dissertação visou explorar a atual situação do sistema previdenciário brasileiro a partir da aplicação de técnicas e estratégias de Ciência de Dados em Dados Abertos Governamentais da Previdência Social. Para isso, foram apresentados dois estudos de caso, objetivando, de forma geral, compreender o impacto socioeconômico que o atual cenário previdenciário nacional causa na população, e simular os efeitos que a reforma prevista pela PEC 06/2019 pode provocar na concessão de benefícios. Com isso, a primeira pesquisa utilizou a metodologia de decomposição do índice de Gini e os microdados disponibilizados pelos censos demográficos de 2000 e 2010 para aferir o impacto que as aposentadorias e pensões causam na distribuição de renda dos estados e municípios brasileiros.

Avaliando os resultados, nota-se que houve um aumento da participação dessa parcela de benefícios na composição de rendimentos da população, e uma diminuição no coeficiente de desigualdade econômica. Todavia, embora as aposentadorias e pensões apresentassem caráter regressivo na distribuição de riquezas, observou-se que os benefícios até um salário mínimo colaboravam para a diminuição da concentração de renda. Tal fator destacou-se nos municípios da região nordeste, que, sobretudo, apresentou os menores índices de médias salariais do Brasil. Além disso, destacam-se as interdependências existentes entre as variáveis analisadas, evidenciando a correlação negativa entre a participação dos benefícios avaliados com o índice de Gini, indicando que ao passo que um aumenta o outro diminui.

Por outro lado, o segundo estudo utilizou os microdados de aposentadorias disponibilizados pela CPI da Previdência Social para simular os efeitos que as novas diretrizes, previstas na PEC 06/2019, causariam nos benefícios de 1995 a 2016 se as mesmas já estivessem em vigor na data em que cada aposentadoria foi concedida. Para isso, foram aplicadas estratégias de ciência de dados para facilitar o armazenamento dos dados, consultas e realização de futuras análises. 

Dessa forma, após validar as informações disponibilizadas, observou-se que 83,28\% das aposentadorias concedidas no período analisado não atendiam às novas regras previstas na reforma. Evidenciou-se, então, que a PEC 06/2019, aliada aos elevados índices de informalidade existentes no Brasil, dificultaria o acesso a tais benefícios, dado a dificuldade que trabalhadores da RGPS tem em manterem-se estáveis em seus empregos. Além disso, estimou-se que aproximadamente 6,48\% dos beneficiários teriam falecido antes de dar entrada nas suas aposentadorias, e que somente 3\% do total receberiam a média integral, ou mais, das suas contribuições como mensalidade.

Embora não tenham sido aplicadas técnicas de inteligência computacional nas análises realizadas, destaca-se a eficiência de utilizar ciência de dados em bases de dados abertas governamentais, uma vez que tal estratégia possibilitou a extração de conhecimento válido acerca de dados sem representatividade, podendo, dessa forma, servir como ferramenta para auxiliar possíveis tomadas de decisão em gestão pública. 

Com objetivo de promover um trabalho colaborativo, foram disponibilizados todos os \textit{scripts}, ferramentas e dados utilizados nessa dissertação na plataforma GitHub, possibilitando esforços da comunidade cientifica para o enriquecimento nas análises realizadas. Com isso, devido ao fato de compreender um campo interdisciplinar, a Previdência Social possibilita que pesquisadores de diversas áreas desenvolvam alternativas para o debate com diferentes óticas, possibilitando, dessa forma, a elaboração políticas públicas cada vez mais justas e eficientes diante do cenário brasileiro.

\section{Dificuldades Encontradas}

Durante o desenvolvimento do trabalho foram encontradas uma série de dificuldades relacionadas, principalmente, a dificuldade de acesso a informações e granularidade dos dados disponibilizados.

Dentre as principais dificuldades encontradas durante a realização deste trabalho, destacam-se:

\begin{itemize}
    \item Ausência de microdados atualizados dos rendimentos de aposentadorias e pensões acerca dos municípios brasileiros, impossibilitando a realização de investigações mais precisas da atual participação que esses benefícios tem na distribuição de renda.
    
    \item Ausência das identificações de estados e municípios nos microdados da CPI da Previdência, impossibilitando análises territoriais mais precisas dos impactos que a PEC 06/2019 pode causar.
    
    \item Ausência das datas de cessação dos benefícios contidos na base da CPI da Previdência, impossibilitando uma avaliação histórica de benefícios ativos.
    
    \item A granularidade das informações disponibilizadas pelo AEPS, divulgando informações agregadas que impossibilitam análises mais detalhadas e complexas.
\end{itemize}


\section{Trabalhos Futuros}
 
Esta pesquisa encontra-se em desenvolvimento, possibilitando a realização de novas análises e discussões acerca da temática abordada. Dessa forma, os seguintes itens estão em processo de aperfeiçoamento e serão publicados em trabalhos futuros:

\begin{itemize}
    \item A exploração complementar da metodologia de decomposição do índice de Gini, efeitos composição e concentração, em outras subcategorias de salários (2, 3 e 4 salários mínimos, por exemplo), para avaliar de forma mais precisão a participação que as aposentadorias e pensões possuem na distribuição de riquezas da população.
    
    \item A aplicação de métodos estatísticos espaciais, como a varredura espacial de Kulldorff, afim de detectar regiões de \textit{clusters} acerca da das variáveis analisadas.
    
    \item A aplicação de técnicas de inteligência computacional, como Redes Bayesianas, para identificar as principais variáveis que influenciam na distribuição de renda da população.
    
    \item Utilizar as taxas de informalidade e desemprego para aferir, de forma mais precisa, o comportamento que as concessões de aposentadorias e pensões teriam se as regras previstas pela PEC 06/2019 já estivessem em vigor.
    
    \item Avaliar os impactos que a reforma da previdência causaria na distribuição de renda da população brasileira utilizando o índice de Gini e outras fontes de dados, como por exemplo a PNAD. 
\end{itemize}

\section{Publicações}

Publicações realizadas durante o desenvolvimento desta dissertação:

\begin{enumerate}[label=(\roman*)]
    \item \textbf{FÉLIX JUNIOR, F. E. A.} ; PEREIRA, A. A. S. ; SANTOS, S. M. ; PUTY, C. A. C. B. ; REGO, L. P. ; SILVA, M. S. Impactos da reforma previdenciária: um estudo acerca das concessões de benefícios. VIII Congresso Internacional Interdisciplinar em Sociais e Humanidades (CONINTER), 2019, UNIT/AL. Anais do(a) Anais do VIII CONINTER. Recife: Even3, 2019. v. 8. p. 1-15.
    
    \item \textbf{FÉLIX JUNIOR, F. E. A.} ; SANTOS, S. M. ; FRANCES, C. R. L. ; MONTEIRO, M. A. ; SILVA, M. S. . Análise da participação das aposentadorias e pensões na distribuição de renda per capita do Brasil - 2000 e 2010. VIII Congresso Internacional Interdisciplinar em Sociais e Humanidades (CONINTER), 2019, UNIT/AL. Anais do(a) Anais do VIII CONINTER. Recife: Even3, 2019. v. 8. p. 1-15.
    
    \item SANTOS, S. M. ; \textbf{FÉLIX JUNIOR, F. E. A.} ; FRANCES, C. R. L. ; REGO, L. P. ; SILVA, M. S. Ciência de dados aplicada em base de dados abertas: uma análise da economia dos municípios brasileiros - 2010 a 2017. VIII Congresso Internacional Interdisciplinar em Sociais e Humanidades (CONINTER), 2019, UNIT/AL. Anais do(a) Anais do VIII CONINTER. Recife: Even3, 2019. v. 8. p. 1-15.
    
    \item OLIVEIRA, E. C. L. ; SANTOS, S. M. ; \textbf{FÉLIX JUNIOR, F. E. A.} ; SILVA, M. S. ; ARAUJO, J. P. L. Identificação paramétrica de moto CC utilizando algoritmos metaheurísticos e mínimos quadrado. XIV Congresso Brasileiro de Inteligência Computacional (CBIC), 2019, Belém. Anais do XIV CBIC, 2019.
    
    \item \textbf{FÉLIX JUNIOR, F. E. A.}; SANTOS, S. M. ; SILVA, M. S. ; CARVALHO, J. C. C. . Utilização de técnicas de machine learning para classificação de sinais de eletrocardiograma. XIV Congresso Brasileiro de Inteligência Computacional (CBIC), 2019, Belém. Anais do XIV CBIC, 2019.
\end{enumerate}