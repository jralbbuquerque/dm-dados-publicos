% ---
% RESUMOS
% ---

% RESUMO em português
\setlength{\absparsep}{18pt} % ajusta o espaçamento dos parágrafos do resumo
\begin{resumo}
A Ciência de Dados trata-se de uma área interdisciplinar relacionada a análise de dados, a qual objetiva a extração de conhecimento e possíveis tomadas de decisão acerca de problemáticas especificas. Nesse contexto, os dados abertos governamentais, que muitas vezes necessitam de pré-tratamentos e métodos computacionais para processamento acerca dos seus conjuntos de dados, apresentam-se como potenciais fontes de informações a serem exploradas na ótica de Ciência de Dados, possibilitando o desenvolvimento de estratégias cada vez mais eficientes e otimizadas em gestão pública. Diante disso, e aliado as recentes discussões relacionadas a reforma na previdência social brasileira, essa dissertação apresenta dois estudos de caso referentes a análises no sistema previdenciário nacional. O primeiro estudo utilizou os microdados referentes aos censos demográficos de 2000 e 2010, disponibilizados pelo IBGE, propondo avaliar a participação que as aposentadorias e pensões possuem na desigualdade de renda da população nos anos avaliados acerca dos estados e municípios brasileiros. Os resultados mostram que, embora os benefícios analisados contribuam para a concentração de renda no Brasil, a parcela correspondente até um salário mínimo contribui para a desconcentração da renda, e aquela acima de um salário contribui para a concentração, sendo um padrão repetitivo em todo o território nacional. Por outro lado, o segundo estudo propôs uma avaliação dos impactos que a reforma da previdência, proposta pela PEC 06/2019, causaria acerca das concessões dos benefícios de aposentadoria entre o período de 1995 a 2016. Observou-se que a PEC 06/2019 dificultaria o acesso aos benefícios, na qual, aproximadamente 83,28\% das aposentadorias não haveriam sido concedidas se a mesma já estivesse em vigor desde 1995.

\textbf{Palavras-chaves}: Ciência de Dados. Previdência Social. Índice de Gini. Dados Abertos. Análise de Dados.
\end{resumo}

% ABSTRACT in english
\begin{resumo}[Abstract]
 \begin{otherlanguage*}{english}
 Data Science is an interdisciplinary area related to data analysis, which aims to extract knowledge and possible decision-making about specific problems. In this context, open government data, which often need pre-treatments and computational methods to process their data sets, present themselves as potential sources of information to be explored taking the Data Science's perspective, allowing the development of strategies each time more efficient and optimized in public management. Given this, and allied to the recent discussions related to the reform in the Brazilian social security, this dissertation presents two case studies referring to analyzes in the national social security system. The first study used the microdata referring to the demographic censuses of 2000 and 2010, made available by IBGE, proposing to evaluate the participation that retirements and pensions have in the income inequality of the population in the years evaluated about Brazilian states and municipalities. The results show that, although the analyzed benefits contribute to the Brazil income concentration, the portion corres=ponding to a minimum wage contributes to the deconcentration of income, and the portion above one salary contributes to the concentration, being a repetitive pattern throughout the country. On the other hand, the second study proposed an evaluation of the impacts caused by the pension reform, which is proposed in PEC 06/2019. It was observed that PEC 06/2019 would hinder access to benefits, in which approximately 83,28\% of the pensions would not have been granted had it been in effect since 1995.

\vspace{\onelineskip}
 
\noindent 
\textbf{Keywords}: Data Science. Social Security. Retirements. Gini Index. Open Data. Data Analysis.
\end{otherlanguage*}
\end{resumo}